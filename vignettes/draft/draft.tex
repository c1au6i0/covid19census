\documentclass[12pt,halfline,a4paper,]{ouparticle}

% Packages I think are necessary for basic Rmarkdown functionality
\usepackage{hyperref}
\usepackage{graphicx}
\usepackage{listings}
\usepackage{color}
\usepackage{fancyvrb}
\usepackage{framed}

%% To allow better options for figure placement
%\usepackage{float}

% Packages that are supposedly required by OUP sty file
\usepackage{amssymb, amsmath, geometry, amsfonts, verbatim, endnotes, setspace}

% For code highlighting I think
\DefineVerbatimEnvironment{Highlighting}{Verbatim}{commandchars=\\\{\}}
\definecolor{shadecolor}{RGB}{248,248,248}
\newenvironment{Shaded}{\begin{snugshade}}{\end{snugshade}}
\newcommand{\AlertTok}[1]{\textcolor[rgb]{0.94,0.16,0.16}{#1}}
\newcommand{\AnnotationTok}[1]{\textcolor[rgb]{0.56,0.35,0.01}{\textbf{\textit{#1}}}}
\newcommand{\AttributeTok}[1]{\textcolor[rgb]{0.77,0.63,0.00}{#1}}
\newcommand{\BaseNTok}[1]{\textcolor[rgb]{0.00,0.00,0.81}{#1}}
\newcommand{\BuiltInTok}[1]{#1}
\newcommand{\CharTok}[1]{\textcolor[rgb]{0.31,0.60,0.02}{#1}}
\newcommand{\CommentTok}[1]{\textcolor[rgb]{0.56,0.35,0.01}{\textit{#1}}}
\newcommand{\CommentVarTok}[1]{\textcolor[rgb]{0.56,0.35,0.01}{\textbf{\textit{#1}}}}
\newcommand{\ConstantTok}[1]{\textcolor[rgb]{0.00,0.00,0.00}{#1}}
\newcommand{\ControlFlowTok}[1]{\textcolor[rgb]{0.13,0.29,0.53}{\textbf{#1}}}
\newcommand{\DataTypeTok}[1]{\textcolor[rgb]{0.13,0.29,0.53}{#1}}
\newcommand{\DecValTok}[1]{\textcolor[rgb]{0.00,0.00,0.81}{#1}}
\newcommand{\DocumentationTok}[1]{\textcolor[rgb]{0.56,0.35,0.01}{\textbf{\textit{#1}}}}
\newcommand{\ErrorTok}[1]{\textcolor[rgb]{0.64,0.00,0.00}{\textbf{#1}}}
\newcommand{\ExtensionTok}[1]{#1}
\newcommand{\FloatTok}[1]{\textcolor[rgb]{0.00,0.00,0.81}{#1}}
\newcommand{\FunctionTok}[1]{\textcolor[rgb]{0.00,0.00,0.00}{#1}}
\newcommand{\ImportTok}[1]{#1}
\newcommand{\InformationTok}[1]{\textcolor[rgb]{0.56,0.35,0.01}{\textbf{\textit{#1}}}}
\newcommand{\KeywordTok}[1]{\textcolor[rgb]{0.13,0.29,0.53}{\textbf{#1}}}
\newcommand{\NormalTok}[1]{#1}
\newcommand{\OperatorTok}[1]{\textcolor[rgb]{0.81,0.36,0.00}{\textbf{#1}}}
\newcommand{\OtherTok}[1]{\textcolor[rgb]{0.56,0.35,0.01}{#1}}
\newcommand{\PreprocessorTok}[1]{\textcolor[rgb]{0.56,0.35,0.01}{\textit{#1}}}
\newcommand{\RegionMarkerTok}[1]{#1}
\newcommand{\SpecialCharTok}[1]{\textcolor[rgb]{0.00,0.00,0.00}{#1}}
\newcommand{\SpecialStringTok}[1]{\textcolor[rgb]{0.31,0.60,0.02}{#1}}
\newcommand{\StringTok}[1]{\textcolor[rgb]{0.31,0.60,0.02}{#1}}
\newcommand{\VariableTok}[1]{\textcolor[rgb]{0.00,0.00,0.00}{#1}}
\newcommand{\VerbatimStringTok}[1]{\textcolor[rgb]{0.31,0.60,0.02}{#1}}
\newcommand{\WarningTok}[1]{\textcolor[rgb]{0.56,0.35,0.01}{\textbf{\textit{#1}}}}

% For making Rmarkdown lists
\providecommand{\tightlist}{%
  \setlength{\itemsep}{0pt}\setlength{\parskip}{0pt}}


% Part for setting citation format package: natbib
\usepackage{natbib}
\bibliographystyle{plainnat}

% Part for setting citation format package: biblatex

% Pandoc header

\begin{document}

\title{covid19census: U.S. and Italy COVID-19 epidemiolagical data joined with
demographic and health related metrics}

\author{%
\name{Claudio Zanettini}\address{Department of Oncology, Johns Hopkins University School of Medicine,
Baltimore, MD, USA}\email{\href{mailto:claudio.zanettini@gmail.com}{claudio.zanettini@gmail.com}}
\and
\name{Luigi Marchionni}\address{Department of Oncology, Johns Hopkins University School of Medicine,
Baltimore, MD, USA}\email{\href{mailto:marchion@jhu.edu}{marchion@jhu.edu}}\thanks{Corresponding author; Email: \href{mailto:marchion@jhu.edu}{marchion@jhu.edu}}
\and
\name{Others to be add}\address{Another University}\email{\href{mailto:otherstobeadd@example.com}{otherstobeadd@example.com}}
}

\abstract{This is the abstract. Claudio's favourite colour is blue, Claudio's
least favirite part of writing a paper is the abstract. Claudio has
postponed till the end the writing of the abstract.}

\date{\today}

\keywords{covid19; R}

\maketitle



\hypertarget{introduction}{%
\section{Introduction}\label{introduction}}

In the mist of a virus pandemic, unraveling the constant flow of
epidemiological data is of paramount importance, not only to guide the
evaluation and implementation of non-pharmacological interventions
(NPI), but also to optimize drug development.

For example, analysis and modeling of COVID-19 confirmed cases and
deaths has been employed, in early phases of the pandemic, to assess the
effects of NPI in China and Europe \citep{flaxman2020, prem2020tlph}.
More recently, the increased flow of COVID-19 data, and the integration
of different sources of information (seasonality of other coronoviruses,
U.S. clinical care) has allowed even more long-term predictions of the
feasibility and effectiveness of possible containment strategies
\citep{kissler2020s}.

Similarly, early evidences of the correlation between Bacille
Calmette-Guérin vaccination and COVID-19 outcomes, spur several clinical
investigations \citep{miller2020m, shet2020m};
\href{https://www.who.int/news-room/commentaries/detail/bacille-calmette-gu\%C3\%A9rin-(bcg)-vaccination-and-covid-19}{WHO}{]}.
However, the implications and conclusions of that initial observation
were curtailed by subsequent models that included more factors
\citep[e.s. age;][]{fukui2020m}. Overall, these few examples underscore
the importance in general, of providing public access to ongoing
COVID-19 metrics and in particular, of including multiple heterogeneous
collections of data in modeling and analysis, to improve predictions and
ultimately address the many challenges of the current pandemic
emergency.

The current \texttt{R} package provides tools to rapidly extract United
States and Italy COVID-19 epidemiological metrics (at county and
regional level, respectively) from different sources and combine them
with other demographic and health related datasets. The goal of the
package is to facilitate multifactorial analysis and modeling of
COVID-19 data by the scientific community.

\hypertarget{alghorithm}{%
\section{Alghorithm}\label{alghorithm}}

A family of \texttt{get} functions is employed by the \texttt{R} package
to dynamically extract updated time-series data from different on-line
sources, combine them and finally return a \texttt{dataframe}.

For \textbf{U.S} the prefix of the functions to extract data is
\texttt{getus\_}, and it is followed by the specific metric of interest:

\begin{itemize}
\tightlist
\item
  \texttt{getus\_covid}: extracts data of COVID-19 from the
  \href{https://github.com/nytimes/covid-19-data}{New York Time github
  repository}.
\item
  \texttt{getus\_dex}: extracts data of DEX, an
  \href{https://github.com/COVIDExposureIndices/COVIDExposureIndices}{activity
  indexes} calculated by Victor Couture, Jonathan Dingel, Allison Green,
  Jessie Handbury, and Kevin Williams based on smartphone movement data
  provided by \texttt{PlaceIQ}.
\item
  \texttt{getus\_tests}: extract info regarding number of tests
  performed, their results and hospitalization from the repository of
  \href{https://covidtracking.com/api\%7D}{the Covid Tracking Project}.
\item
  \texttt{getus\_all}: executes all the above functions and join the
  results with other datasets statically contained in the package, and
  returns a \texttt{dataframe} with 304 variables.
\end{itemize}

Data regarding the household composition, population sex and age and
poverty levels (2018), were retrieved from the
\href{https://data.census.gov/cedsci/table?q=United\%20States}{American
Community Survey}. Medical conditions, tobacco use, cancer and, data
relative to the number of medical and emergency visits (2017) of
medicare beneficiaries were obtained from the
\href{https://data.cms.gov/mapping-medicare-disparities}{Mapping
Medicare Disparities}. The number of hospital beds per county (2020) was
calculated from data of the
\href{https://hifld-geoplatform.opendata.arcgis.com/datasets/hospitals/data?page=18}{Homeland
Infrastructure Foundation}.

For \textbf{Italy}, the prefix of the function is \texttt{getit\_}
followed by \texttt{covid} or \texttt{all}.

\begin{itemize}
\tightlist
\item
  \texttt{getit\_covid}: extracts data of COVID-19 cases, deaths,
  hospitalizations and tests from the
  \href{\%22https://raw.githubusercontent.com/pcm-dpc/COVID-19/master/dati-regioni/dpc-covid19-ita-regioni.csv\%22}{Protezione
  Civile}.
\item
  \texttt{getit\_all}: executes the above function, join the results
  with other datasets statically contained in the package and returns a
  \texttt{dataframe} with 64 variables.
\end{itemize}

Age and sex of the population (2019), first aid and medical guard visits
(2018), smoking status (2018), prevalence of chronic conditions (2018),
annual-household income (2017), household crowding index (2018) and
body-mass index were collect from
\href{http://dati.istat.it/?lang=en}{ISTAT}. Prevalence of types of
cancer patients (2016), influenza-vaccination coverage (2019) and the
number of hospital beds per 1000 people (2017) were obtained from
\href{http://www.dati.salute.gov.it/}{Ministero della Salute}. Data of
particulate 2.5 (2017) comes from the
\href{https://annuario.isprambiente.it/pon/basic/14}{Istituto Superiore
Per La protezione Ambientale}.

The package documentation reports and describes each variable
(\texttt{colnames}) and lists all the data sources of each of the
functions. Because of the large amount of variables and, in order to
facilitate exploration of the documentation, it was deemed more
practical to create separate functions with separate documentation for
each of the country.

\hypertarget{implementation-and-use}{%
\section{Implementation and use}\label{implementation-and-use}}

The package is current available on
\href{https://github.com/c1au6i0/covid19census}{github}. The following
code launch the functions and assign the returned \texttt{dataframes} to
different names.

\bigskip

\begin{Shaded}
\begin{Highlighting}[]
\KeywordTok{library}\NormalTok{(covid19census)}
\NormalTok{dat_it <-}\StringTok{ }\KeywordTok{getit_all}\NormalTok{()}
\end{Highlighting}
\end{Shaded}

\begin{verbatim}
## Italy COVID-19 data up to 2020-04-26 17:00:00 successfully retrived!
\end{verbatim}

\begin{Shaded}
\begin{Highlighting}[]
\NormalTok{dat_us <-}\StringTok{ }\KeywordTok{getus_all}\NormalTok{()}
\end{Highlighting}
\end{Shaded}

\begin{verbatim}
## US COVID-19 data up to 2020-04-26 successfully retrived!
\end{verbatim}

\begin{verbatim}
## US mobility data up to 2020-04-17 successfully retrived!
\end{verbatim}

\begin{verbatim}
## US test data up to 2020-04-26 successfully retrived!
\end{verbatim}

\begin{Shaded}
\begin{Highlighting}[]
\KeywordTok{unlist}\NormalTok{(}\KeywordTok{lapply}\NormalTok{(}\KeywordTok{list}\NormalTok{(dat_it, dat_us), class))}
\end{Highlighting}
\end{Shaded}

{[}1{]} ``data.frame'' ``data.frame''

\bigskip

Information on the dataframes generated by the two functions are
reported in table below {[}table \ref{tab:tab_dat}{]}.

\bigskip

\begin{table}[ht]
\centering
\begin{tabular}{ccc}
  \hline
 & getus\_all & getit\_all \\ 
  \hline
columns & 304 & 64 \\ 
  counties-regions & 2790 & 21 \\ 
  sources & 7 & 4 \\ 
  from & 2020-01-21 & 2020-02-24 \\ 
   \hline
\end{tabular}
\caption{Dataframes retuned by the functions.
    The table reports number of columns, number of unique regions (Italy) and counties (U.S.),
    unique sources of data was scarped and earliest data related to COVID-19 metrics, of the dataframes returned
    by the two functions.
    } 
\label{tab:tab_dat}
\end{table}

Data exploration and modeling can be conveniently performed on a
(single) dataframe that contains COVID-19 as well as many other metrics
retrieved from multiple sources. For example, in {[}figure
\ref{fig:fig_corr}{]} is reported a correlation analysis of pair of
selected U.S. variables.

\begin{figure}[p]
\includegraphics[width=1\linewidth]{draft_files/figure-latex/fig_corr-1} \caption{United States Correlation Matrix. An example of exploratory analysis of data from different sources combined by the function `getus\_all`. Colours indicates Pearson's correlation between pairs of variables}\label{fig:fig_corr}
\end{figure}

\hypertarget{conclusions}{%
\section{Conclusions}\label{conclusions}}

The \texttt{R} package \texttt{covid19census} extracts and integrates
epidemiological COVID-19 data (Italy and U.S at the regional and county
level, respectively) with several other demographic and health related
indexes. By combining data form different sources, the package is aimed
at promoting and simplifying the analysis and modeling of COVID-19 data.


\begin{notes}[Acknowledgements]
\emph{Funding} : L.M. was supported by NIH-NCI grants P30CA006973,
U01CA196390, R01CA200859and the Department of Defense (DoD) office of
the Congressionally Directed Medical Research Programs (CDMRP) award
W81XWH-16-1-0739.

\emph{Conflict of interest} : none.
\end{notes}


\renewcommand\refname{References}

\bibliography{mybibfile.bib}



\end{document}
